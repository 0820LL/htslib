\documentclass[10pt]{article}
\usepackage{color}
\definecolor{gray}{rgb}{0.7,0.7,0.7}
\usepackage{framed}
\usepackage{enumitem}
\usepackage{longtable}

\addtolength{\textwidth}{3.4cm}
\addtolength{\hoffset}{-1.7cm}
\addtolength{\textheight}{4cm}
\addtolength{\voffset}{-2cm}

\makeindex

\begin{document}

In BCF2, a typed value consists of a typing byte and the actual value with type
mandated by the typing byte. In the typing byte, The lowest four bits (bit
0--3) give the atomic type. The 7-th bit is set if the value is a vector. For a
vector, if bits 4--6 are not all set, the 3-bit number gives the number of
elements in the vector; otherwise, a typed integer following the typing byte
gives the vector size.
\begin{center}
{\small\begin{tabular}{rlrl}
\hline
Bit 0--3 & C type & Missing value & Description \\
\hline
1 & {\tt int8\_t}   & {\tt 0x80}               & signed 8-bit integer \\
2 & {\tt int16\_t}  & {\tt 0x8000}             & signed 16-bit integer \\
3 & {\tt int32\_t}  & {\tt 0x80000000}         & signed 32-bit integer \\
4 & {\tt int64\_t}  & {\tt 0x8000000000000000} & signed 64-bit integer \\
5 & {\tt float}     & {\tt 0x7F800001}         & IEEE 32-bit floating pointer number \\
6 & {\tt double}    & {\tt 0x7FF0000000000001} & IEEE 64-bit floating pointer number \\
7 & {\tt char}      & `{\tt \char92 0}'        & character \\
8 & {\tt uint8\_t}  &                    & boolean, 0 or 1 only \\
9 & {\tt char*}     & ``{\tt\char92 0}'' & null-terminated C string \\
10 & {\tt uint8\_t} &                    & index of the string in the dictionary \\
11 & {\tt uint16\_t}&                    & index of the string in the dictionary \\
12 & {\tt uint32\_t}&                    & index of the string in the dictionary \\
13 & {\tt uint8\_t} &                    & unsigned 8-bit integer \\
\hline
\end{tabular}}
\end{center}

A genotype is encoded in a typed vector with the number of elements equal to
the maximum ploidy among all samples at a site. For one individual, each allele
is encoded as follows. The lowest bit is set if the allele has been phased.
The integer encoded in the higher bits gives the allele. The integer equals
0 if the allele in unknown; equals 1 if the haplotype is absent (e.g. at an
X chromosome site of a male); equals 2 for a reference allele; equals an
integer larger than 2 for an alternate allele. For a more compact
representation of a diploid genotype with no more than 6 alleles, we may
encode a genotype using a single unsigned 8-bit integer (type 13). In this
case, the higher 4 bits encode the allele on the first haplotype and the lower
encode on the second haplotype.

\begin{table}[h]
\centering
{\small
\begin{tabular}{|l|l|l|p{8.0cm}|l|r|}
  \cline{1-6}
  \multicolumn{3}{|c|}{\bf Field} & \multicolumn{1}{c|}{\bf Description} & \multicolumn{1}{c|}{\bf Type} & \multicolumn{1}{c|}{\bf Value} \\\cline{1-6}
  \multicolumn{3}{|l|}{\sf magic} & BCF2 magic string & {\tt char[4]} & {\tt BCF\char92 2}\\\cline{1-6}
  \multicolumn{3}{|l|}{\sf l\_text} & Length of the header text, including any {\tt NULL} padding & {\tt uint32\_t} & \\\cline{1-6}
  \multicolumn{3}{|l|}{\sf text} & Plain VCF header text; not necessarily {\tt NULL} terminated & {\tt char[{\sf l\_text}]} & \\\cline{1-6}
  \multicolumn{6}{|c|}{\textcolor{gray}{\it List of VCF records (until the end of the BGZF section)}} \\\cline{2-6}
  & \multicolumn{2}{l|}{\sf block\_size} & Length of the remainder of the VCF record & {\tt uint32\_t} & \\\cline{2-6}
  & \multicolumn{2}{l|}{\sf CHROM} & Reference sequence ID & {\tt int32\_t} & \\\cline{2-6}
  & \multicolumn{2}{l|}{\sf POS} & 0-based leftmost coordinate & {\tt int32\_t} & \\\cline{2-6}
  & \multicolumn{2}{l|}{\sf ID} & {\sf NULL} terminated string; field length inferred & {\tt char*} & [``{\tt\char92 0}'']\\\cline{2-6}
  & \multicolumn{2}{l|}{\sf REF} & {\sf NULL} terminated string; field length inferred & {\tt char*} & \\\cline{2-6}
  & \multicolumn{2}{l|}{\sf n\_alt} & Number of {\sf ALT} alleles & {\tt int16\_t} & \\\cline{2-6}
  & \multicolumn{2}{l|}{\sf ALT} & {\sf NULL} terminated string; alleles separated by {\sf NULL} & {\tt char*} & \\\cline{2-6}
  & \multicolumn{2}{l|}{\sf QUAL} & Variant quality. {\tt 0x7F800001} for a missing value. & {\tt float} & [0x7F800001] \\\cline{2-6}
  & \multicolumn{2}{l|}{\sf FILTER} & list of filters; filters are defined in the dictionary & {\tt typed vec} & [0x8A] \\\cline{2-6}
  & \multicolumn{2}{l|}{\sf n\_info} & number of info key-value pairs & {\tt typed int} & \\\cline{2-6}
  & \multicolumn{5}{c|}{\textcolor{gray}{\it List of key-value pairs in the INFO field (n=n\_info)}} \\\cline{3-6}
  & & {\sf info\_key} & Key, defined in the dictionary & {\tt typed int} & \\\cline{3-6}
  & & {\sf info\_value} & Value & {\tt typed val} &\\
  \cline{2-6}
  & \multicolumn{2}{l|}{\sf n\_fmt} & number of format fields & {\tt typed int} & \\\cline{2-6}
  & \multicolumn{5}{c|}{\textcolor{gray}{\it List of formats and sample information (n=n\_fmt)}} \\\cline{3-6}
  & & {\sf fmt\_key} & Format key, defined in the dectionary & {\tt typed int} & \\\cline{3-6}
  & & {\sf fmt\_type} & Typing byte of each individual value, possibly followed by a typed int for the vector length & {\tt uint8\_t+} & \\\cline{3-6}
  & & {\sf fmt\_value} & Array of values. The information of each individual is concatenated in the array. Every value is of the same {\sf fmt\_type}. & (by {\sf fmt\_type}) &\\
  \cline{1-6}
\end{tabular}}
\end{table}


\end{document}
